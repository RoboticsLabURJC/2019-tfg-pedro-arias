\begin{minted} [frame=lines, framesep=2mm, baselinestretch=1.2, bgcolor=LightGray, fontsize=\footnotesize, linenos]{python}

    def calc_path(self):
        # Fija los parámetros iniciales del escaneo
        self.set_scan(self.p_init)

        # Primera pasada
        self.wayp.append(self.v_scan)
        self.calc_intermediates(self.v_scan, 
                    self.get_first_vertex(self.v_scan, self.l_scan))
        self.wayp.append(self.get_first_vertex(self.v_scan, self.l_scan))

        i = 1
        while True:
            # Calcula la siguiente pasada
            aux = self.get_next_aux(i) 
            # Comprueba si la pasada es válida
            vertexes_valid = self.get_valid_vertex(aux) 
            # Ordena los puntos de la pasada
            vertexes_valid = self.get_next_vertex(i, vertexes_valid, self.u_scan)

            if len(vertexes_valid) == 0: # FIN
                break
            elif len(vertexes_valid) == 1:
                self.wayp.append(vertexes_valid[0])
            elif len(vertexes_valid) >= 2:
                self.wayp.append(vertexes_valid[0]) # Siguiente pasada
                self.calc_intermediates(vertexes_valid[0], vertexes_valid[-1])
                self.wayp.append(vertexes_valid[-1])

            i = i + 1

\end{minted}