\documentclass[../main.tex]{subfiles}

Este capítulo reúne las conclusiones extraídas tras la realización del proyecto y describe posibles líneas futuras de trabajo que surgen a partir del mismo. Estas líneas de trabajo suponen una vía de desarrollo para la elaboración de futuras versiones de la aplicación.

\section{Conclusiones} \label{section:concl-concl}
El desarrollo concluye en una nueva versión de UAVCommander. Estos meses de trabajo han permitido que la aplicación haya avanzado desde un punto inicial de desarrollo hasta en un estado actual de prototipo precomercial. Esta evolución se puede observar en la lista de reproducción \footnote{https://www.youtube.com/playlist?list=PLGlX46StCA-QAXmeQp5omNhllGJ3CMjcO} donde se han subido los vídeos con las diferentes versiones y funciones que se han ido lanzando. Los vídeos demuestran como los objetivos descritos durante la Sección \ref{section:intro-objetivos} se han satisfecho. \\
El producto final desarrollado consiste en una aplicación con cerca de 6000 líneas de código en lenguaje Python dividido en diferentes ficheros. La aplicación se ha desarrollado siguiendo un control de versiones y una metodología en espiral con incidencias y parches. Además, el procedimiento seguido incluye validaciones semanales con los tutores. \\
El objetivo principal era el desarrollo software de una estación tierra. A lo largo de los capítulos de esta memoria se han explicado las diferentes funcionalidades y logros conseguidos. Si bien es cierto que la aplicación se encuentra en fase de maduración, ya permite operar con aeronaves reales, tanto con ala fija como con ala rotatoria. Son varios los vuelos de prueba que se han realizado, como se ha podido observar a los largo del capítulo de validación experimental (Cap. \ref{chapter:experim}). Por ello, se concluye que el objetivo principal ha sido ampliamente satisfecho con un software funcional que sirve de base para futuras versiones que incluyan otras mejoras. \\
Entre los objetivos secundarios se encontraban una serie de requisitos funcionales. Estas funciones también han sido satisfechas:

\begin{itemize}
    \item El primer objetivo secundario era dotar a la aplicación de una caracterización de la aeronave y carga de pago. Para dar soporte a esta funcionalidad, se ha desarrollado una pestaña dentro de la interfaz gráfica de usuario que recoge los diferentes parámetros de configuración. Además, se ha dotado a la aplicación de un sistema de carga y guardado en ficheros para la reutilización de configuraciones. Esto agiliza mucho su uso, pues el usuario solo realizaría la caracterización una vez y podría reutilizar el archivo de configuración entre distintas ejecuciones de la aplicación.
    \item El segundo objetivo secundario consistía en poder crear misiones automáticas desde la aplicación. En la segunda pestaña de la aplicación se incluyen un sistema de mapas y un creador de misiones para dar soporte a esta funcionalidad. Son varios los mapas soportados, como se ha explicado durante la Sección \ref{section:dis-map}. Además, las misiones creadas pueden ser de dos tipos, misiones polilínea y misiones por patrón. La creación de misiones y sus tipos se explica a lo largo de la Sección \ref{section:dis-mision}.
    \item El tercer objetivo secundario recogía la necesidad de disponer de una lista de comprobaciones de seguridad previas al vuelo. Esta lista se encuentra en la tercera pestaña de la aplicación.
    \item El cuarto objetivo secundario proponía un sistema de seguimiento sobre la aeronave en vuelo. La cuarta pestaña de la aplicación cumplimenta este requisito. El seguimiento permite visualizar la posición de la aeronave sobre el mapa en todo momento junto con una serie de datos de telemetría y navegación.
    \item Finalmente, el quinto objetivo secundario incluye la posibilidad de un análisis de datos post-vuelo. La quinta pestaña de la aplicación proporciona este servicio, donde los logs de vuelo se pueden descargar y eliminar.
\end{itemize}

Las diferentes funcionalidades secundarias se ven cumplimentadas en cada una de las pestañas de la aplicación. Por tanto, se concluye que los requisitos funcionales se ven cumplimentados dentro del conjunto de la aplicación. \\
Por último, la aplicación se ha diseñado e implementado en todo momento primando la simplicidad y reduciendo al mínimo la interacción con el operario. Con esta visión se pretende facilitar el aprendizaje y el uso de la aplicación, junto con la reducción de los errores que puedan surgir por motivo de equivocaciones humanas. Además, se ha desarrollado siguiendo la línea de aplicaciones existentes similares (como \emph{QGroundControl} o \emph{MissionPlanner}), pero manteniendo la sencillez allí donde las principales soluciones del mercado pecan al incluir multitud de opciones y configuraciones. Así pues, el operario solo interacciona con la aplicación cuando es estrictamente necesario, y los automatismos se tratan de llevar a máximos. \\

\section{Líneas futuras} \label{section:concl-fut}
Este trabajo desarrolla un prototipo funcional de una estación de tierra. Son muchos los aspectos a mejorar y las funciones a incluir que se han detectado a lo largo del desarrollo. A continuación se incluyen una lista de posibles mejoras futuras:

\begin{itemize}
    \item Descarga de teselas del mapa mediante multihilo para agilizar la velocidad de navegación a través del mapa.
    \item Aumentar el número de comprobaciones sobre una misión creada para aumentar la seguridad de la misma. Algún ejemplo podría ser: comprobaciones de radio mínimo de giro, comprobaciones en la altitud del terreno sobre la ruta a realizar, etc.
    \item Añadir nuevos tipos de misión como misiones de corredor.
    \item Permitir un tamaño variable de la aplicación en pantalla, lo que permitiría modificar libremente el tamaño de la ventana. Para ello, sería necesario incluir un tamaño de teselas variable.
    \item Modificación dinámica de los puntos de paso que permita cambiar ciertos valores  como el tipo de punto, el orden, la altura, etc.
    \item Habilitar funciones de \emph{deshacer} o \emph{rehacer} durante la creación de misiones. Mitigar así los errores introducidos al añadir puntos.
    \item Actualización automática de los puntos visibles sobre la lista de puntos.
    \item Mejorar visualmente la aplicación. Substituir texto por iconos, habilitar modo oscuro, etc.
    \item Internacionalización, dar soporte a la aplicación en otros idiomas como portugués, francés, etc.
\end{itemize}
