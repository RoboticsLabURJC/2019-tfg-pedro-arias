\begin{abstract}
    Los robots aéreos forman parte de nuestro día a día con aplicaciones en muchos ámbitos de nuestra vida. Este trabajo proporciona una solución software para la operación automática y remota de aeronaves. Esta solución consiste en una aplicación de escritorio tipo estación terrestre que permite la programación de misiones para robots aéreos. \\
    La aplicación, UAVCommander, se ha diseñado e implementado en lenguaje Python y utiliza MAVLink como protocolo de comunicaciones. Entre las funciones del software destacan la especificación de misiones de navegación polilínea (secuencia de puntos de paso) así como de misiones por patrón, en las que se especifica un área a barrer, en forma de polígono, y la aplicación calcula automáticamente la ruta a seguir de modo que se recorra todo ese área. Además incluye la caracterización paramétrica de la aeronave, la comprobación de requisitos antes de cargar esas misiones en la aeronave y el seguimiento en vuelo de la ejecución de esas misiones junto a datos de navegación. \\
    La aplicación programada se ha validado experimentalmente, tanto con drones simulados como con una aeronave real en un campo de vuelo. Además de las pruebas unitarias, el sistema final ha funcionado satisfactoriamente en las pruebas integrales.
\end{abstract}

\keywords{Estación de tierra, UAVS, robótica, robótica aérea, drones, MAVLink.}