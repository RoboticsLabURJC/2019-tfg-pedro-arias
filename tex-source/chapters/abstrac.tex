\begin{abstract}
    Este trabajo proporciona una solución software para la operación automática y remota de aeronaves. Esta solución consisten en una aplicación de escritorio tipo estación terrestre que permite la programación de misiones para robots aéreos. \\
    El documento recoge la infraestructura utilizada durante el desarrollo de la aplicación. Esta infraestructura se compone de diferentes herramientas software y hardware que se explican durante el texto. \\
    El diseño se realiza teniendo en cuenta en todo momento los objetivos del proyecto. Junto al diseño se detalla la implementación seguida para cumplir tal especificaciones.
    Los diferentes bloques en los que se organiza el código, los hilos de ejecución, el sistema de ventanas y las interfaces utilizadas se incluyen con el diseño e implementación. \\
    Además, con el objetivo de reflejar los resultados obtenidos, se proporciona un manual de usuario y una lista de casos de uso que exponen una visión global de la aplicación. El manual de usuario muestra la apariencia y las funcionalidades del software mientras que los casos de uso indican los procedimientos a realizar por el usuario para completar determinadas tareas. \\
    Por último, se aportan una serie de validaciones experimentales para demostrar el funcionamiento de la estación terrestre desarrollada. Estos experimentos en real dan credibilidad e introducen las conclusiones y la valoración de los objetivos iniciales, junto a líneas futuras de mejora y desarrollo. \\
\end{abstract}

\keywords{Estación de tierra, UAVS, robótica, robótica aérea, drones, MAVLink.}